\documentclass[12pt]{article}
\usepackage[utf8]{inputenc}
\usepackage{float}
\usepackage{amsmath}
\usepackage{amssymb}

\usepackage[hmargin=3cm,vmargin=6.0cm]{geometry}
%\topmargin=0cm
\topmargin=-2cm
\addtolength{\textheight}{6.5cm}
\addtolength{\textwidth}{2.0cm}
%\setlength{\leftmargin}{-5cm}
\setlength{\oddsidemargin}{0.0cm}
\setlength{\evensidemargin}{0.0cm}

%misc libraries goes here
\usepackage{fitch}

\begin{document}

\section*{Student Information } 
%Write your full name and id number between the colon and newline
%Put one empty space character after colon and before newline
Full Name : Mithat Can Timurcan\\
Id Number :  2581064\\

% Write your answers below the section tags
\section*{Answer 1}
\paragraph*{a)}
\begin{itemize}
            \item Assume that the set $C \subseteq \mathbb{R}^n $ is a convex set.
            \item Since $C$ is a convex set, we have the following:
            \begin{equation*}
                \begin{split}
                 \forall x_1, x_2 \in C, t \in [0,1] \quad tx_1+(1-t)x_2\in C \\
                \end{split}
            \end{equation*}
            \item We're going to use mathematical induction to prove that for a fixed $m > 3$, any linear combination of m points in the set C is also in the set C.
            \item \textbf{Base Case ($m = 3$):}
            \begin{itemize}
                \item Consider three points $x_1, x_2, x_3$ in the convex set $C$. We want to prove that for any non-negative weights $\lambda_1, \lambda_2, \lambda_3$ such that $\sum_{i=1}^{3} \lambda_i = 1$, the convex combination $\sum_{i=1}^{3} \lambda_i x_i$ is also in $C$.
                \item Since \(C\) is convex, the line segment between any two points in \(C\) lies in \(C\). We can use this property to prove the base case.
                \item Consider the convex combination \(\lambda_1 x_1 + \lambda_2 x_2\). Without loss of generality, let's look at the combination of the first two points:
                \begin{equation*}
                    \begin{split}
                        \lambda_1 x_1 + \lambda_2 x_2
                    \end{split}
                \end{equation*}
                \item Since \(\lambda_1 + \lambda_2 = 1\), this is a convex combination of \(x_1\) and \(x_2\). By the convexity of \(C\), this combination is in \(C\).
                \item Now, consider the combination of this result with the third point:
                \begin{equation*}
                    \begin{split}
                        \lambda_3 (\lambda_1 x_1 + \lambda_2 x_2) + (1 - \lambda_3) x_3 = \lambda_3 \lambda_1 x_1 + \lambda_3 (1-\lambda_1) x_2 + (1-\lambda_3)x_3
                    \end{split}
                \end{equation*}
                \item Since $\lambda_3 \lambda_1 + \lambda_3(1-\lambda_1) + (1-\lambda_3) = 1$, this is a convex combination of $\lambda_1 x_1 + \lambda_2 x_2$ and $x_3$. Therefore, it is also in $C$.
                \item Hence, we've proved that the base case holds.
            \end{itemize}
            \item \textbf{Inductive Step ($m = 3$):}
            \begin{itemize}
                \item Assume the property holds for some \(k\) (i.e., for any \(k\) points in \(C\), their convex combination is in \(C\)). We want to prove it for \(k+1\).
                \item Consider \(k+1\) points \(x_1, x_2, \ldots, x_{k+1}\) in \(C\). We can use the inductive hypothesis on the first \(k\) points:
                \[
                \sum_{i=1}^{k} \lambda_i x_i \in C
                \]
                \item Now, for the \(k+1\)-th point \(x_{k+1}\), we can use the base case (\(m=3\)):
                \[
                \lambda_{k+1} x_{k+1} + (1 - \lambda_{k+1})\left(\sum_{i=1}^{k} \lambda_i x_i\right)
                \]
                \item Since \(\lambda_{k+1} \geq 0\) and \(\sum_{i=1}^{k+1} \lambda_i = 1\), this is a convex combination of \(x_{k+1}\) and \(\sum_{i=1}^{k} \lambda_i x_i\). By the base case, this convex combination is in \(C\). Therefore, the property holds for \(k+1\).
            \end{itemize}
            \item By mathematical induction, we've proved that for a fixed $m > 3$, any linear combination of m points in the set C is also in the set C.
\end{itemize}{}
\paragraph*{b)}
\begin{itemize}
            \item Assume that the functions $f$ and $g$ are convex functions.
            \item Since $f$ and $g$ are a convex functions, we have the following:
            \begin{equation*}
                \begin{split}
                    \forall x_1, x_2 \in \mathbb{R}^n, t \in [0, 1] \quad f(tx_1 + (1 - t)x_2) \le tf(x_1) + (1 - t)f (x_2)\\
                    \forall x_1, x_2 \in \mathbb{R}^n, t \in [0, 1] \quad g(tx_1 + (1 - t)x_2) \le tg(x_1) + (1 - t)g (x_2)\\
                \end{split}
            \end{equation*}
            \item Let's consider the functions: 
            \begin{equation*}
                \begin{split}
                    f(x) = x^4, \ g(x) = x^2 - 4 \text{ and their composition } h(x) = f(g(x)) = (x^2-4)^4\\
                \end{split}
            \end{equation*}
            \item Pick $x_1 = 1/3$ , $x_2 = 1/4$ and $t = 0.5$:
            \begin{equation*}
                \begin{split}
                    h(tx_1+(1 - t)x_2) = h(0.5 \times 1/3 + 0.5 \times 1/4)=h(7/48)=((7/48)^2 - 1)^4\\
                    th(x_1)+(1-t)h(x_2)=0.5((1/3)^2-1) + 0.5((1/4)^2-1)\\
                \end{split}
            \end{equation*}
            \item Comparing these results:
            \begin{equation*}
                \begin{split}
                    ((7/48)^2 - 1)^4 \approx 0.9176\\
                    0.5((1/3)^2-1)^4 + 0.5((1/4)^2-1)^4 \approx 0.6983\\
                \end{split}
            \end{equation*}
            \item Since $((7/48)^2 - 1)^4 > 0.5((1/3)^2-1)^4 + 0.5((1/4)^2-1)^4$, the inequality in the definition doesn't hold. Hence while $f$ and $g$ are convex functions, their composite function $f(g(x))$ may not be convex.
\end{itemize}{}
\paragraph*{c)}
\begin{itemize}
            \item Let A function $f(.) : S \subseteq \mathbb{R}^n \rightarrow \mathbb{R}$ is a convex function be $s$ and $S$ is convex set and the function
            $g(t) = f (x + tv)$ is a convex function for all $t \in \mathbb{R}$ such that $x + tv \in S$ be $r$.
            \item We need to show that $s\rightarrow r$ and $r \rightarrow s$, then we're done.
            \item $s \rightarrow r$.
            \begin{itemize}
                \item Assume that  $f(.) : S \subseteq \mathbb{R}^n \rightarrow \mathbb{R}$ is a convex function.
                \item If $f(.)$ is a convex function and defined on S, then S must be a convex set since the domain of a convex function is also a convex set by the definition of convexity.
                \item Now let's consider $g(x) = f(x+tv)$. We need to show that $g(t)$ is convex $\forall t$ such that $x + tv \in S$.
                \item Let $y_1 = x + t_1v$ and $y_2 = x + t_2v$ be two points in $S$ where $t_1,t_2$ such that $y_1, y_2 \in S$.
                \item Consider $z = \lambda y_1 + (1 - \lambda)y_2$, where $\lambda$ is a convex linear combination coefficient. ($\lambda \in [0,1]$)
                \begin{equation*}
                    \begin{split}
                        z = \lambda (x + t_1v) + (1-\lambda)(x+t_2v)\\
                        z = x + (\lambda t_1 + (1-\lambda) t_2)v\\
                    \end{split}
                \end{equation*}
                \item Since $S$ is a convex set, $x + (\lambda t_1 + (1-\lambda) t_2)v \in S$ and by the convexity of $f(.)$, we can say that $g(t) = f(x + tv)$ is also convex.
                \item $s \rightarrow r$ has been proved.
            \end{itemize}
            \item $r \rightarrow s$.
            \begin{itemize}
                \item Assume that $S$ is a convex set and $g(t) = f(x + tv)$ is a convex function for all $t \in \mathbb{R}$ such that $x + tv \in S$. 
                \item In order to show $f(.)$ is a convex function, we need to consider two arbitrary points $x_1,x_2$ in the domain of f and show the inequality $f(\lambda x_1 + (1-\lambda)x_2) \leq \lambda f(x_1) + (1-\lambda)f(x_2)$ holds for $\forall \lambda \in [0,1]$.
                \item Consider $x_1, x_2$ in the domain of $f(.)$. Let $\lambda$ be a convex linear combination coefficient ($\lambda \in [0,1]$).
                \item Now, let's consider $z = \lambda x_1 + (1-\lambda)x_2$
                \item Since $S$ is convex, by the definition we can say that $z \in S$. Therefore we can also use the convexity of $g(t) = f(x + tv)$ for $t$ such that $x+tv=z$.
                \begin{equation*}
                    \begin{split}
                        g(t) = f(x+tv) = f(\lambda x_1 + (1-\lambda)x_2)\\
                    \end{split}
                \end{equation*}
                \item And by the convexity of $g(t)$ we can obtain:
                \begin{equation*}
                    \begin{split}
                        f(\lambda x_1 + (1-\lambda)x_2) \leq \lambda f(x_1) + (1-\lambda)f(x_2)\\
                    \end{split}
                \end{equation*}
                \item Therefore, $r \rightarrow s$ is also proved.
            \end{itemize}
\end{itemize}{}

\section*{Answer 2}
\paragraph*{a)}
\begin{itemize}
    \item (i) where $X$ is uncountable.
    \begin{itemize}
        \item Let's denote this set as $\Sigma$.
        \item Consider X as the set of real numbers. Since $X-X = \emptyset$, we can say that $X \in \Sigma$. However, it's complement $\emptyset$ is not in $\Sigma$ since $X-\emptyset = X$ and $X$ is not finite.
        \item Therefore, the given set $\Sigma$ is not $\sigma$-algebra on $X$ when $X$ is uncountable.
    \end{itemize}
    \item (ii) where $X$ is countably infinite.
    \begin{itemize}
        \item Let's denote this set as $\Sigma$.
        \item Consider X as the set of natural numbers. Since $X-X = \emptyset$, we can say that $X \in \Sigma$. However, it's complement $\emptyset$ is not in $\Sigma$ since $X-\emptyset = X$ and $X$ is not finite.
        \item Therefore, the given set $\Sigma$ is not $\sigma$-algebra on $X$ when $X$ is countably infinite.
    \end{itemize}
    \item (iii) where $X$ is finite.
    \begin{itemize}
        \item Let's denote this set as $\Sigma$ and check the conditions.
        \item $X$ is in $\Sigma$ since $X-X=\emptyset$ is finite and $\emptyset$ is also in $\Sigma$ since $X-\emptyset = X$ is finite.
        \item If $A$ is in $\Sigma$, then $X-A$ is either finite or $\emptyset$. The complement of $X-A$ is A, which is also a finite set in $\Sigma$.
        \item If $A_1,A_2,...$ are in $\Sigma$, then their complements $X-A_1,X-A_2$, are finite or $\emptyset$. The union of these sets corresponds to the complement of the union $A=A_1\cup A_2\cup...$, which is also in $\Sigma$.
        \item Therefore, the given set $\Sigma$ is a $\sigma$-algebra on $X$ when $X$ is finite.
    \end{itemize}
\end{itemize}
\paragraph*{b)}
\begin{itemize}
    \item (i) where $X$ is uncountable.
    \begin{itemize}
     \item Let's denote this set as $\Sigma$.
        \item Consider X as the set of real numbers and $U = \mathbb{R} - \mathbb{N}$. Since $X-U = \mathbb{N}$, we can say that $U \in \Sigma$. However, it's complement $\mathbb{N}$ is not in $\Sigma$ since $X-\mathbb{N} = \mathbb{R} - \mathbb{N}$ and $\mathbb{R} - \mathbb{N}$ is uncountable.
        \item Therefore, the given set $\Sigma$ is not $\sigma$-algebra on $X$ when $X$ is uncountable.
    \end{itemize}

    \item (ii) where $X$ is countably infinite.
    \begin{itemize}
        \item Let's denote this set as $\Sigma$ and check the conditions.
        \item $X$ is in $\Sigma$ since $X-X=\emptyset$ is countable and $\emptyset$ is also in $\Sigma$ since $X-\emptyset = X$ is countable.
        \item If $A$ is in $\Sigma$, then $X-A$ is either countable or is all of $X$. The complement of $X-A$ is A, which is also a countable set in $\Sigma$ since $A\subseteq X$.
        \item If $A_1,A_2,...$ are in $\Sigma$, then their complements $X-A_1,X-A_2$, are either countable or are all of $X$. The union of these sets corresponds to the complement of the union $A=A_1\cup A_2\cup...$, which is also in $\Sigma$ since union of countable sets is also countable.
        \item Therefore, the given set $\Sigma$ is a $\sigma$-algebra on $X$ when $X$ is countably infinite.
    \end{itemize}
    \item (iii) where $X$ is finite.
    \begin{itemize}
        \item Let's denote this set as $\Sigma$ and check the conditions.
        \item $X$ is in $\Sigma$ since $X-X=\emptyset$ is countable and $\emptyset$ is also in $\Sigma$ since $X-\emptyset = X$ and $X$ is countable.
        \item If $A$ is in $\Sigma$, then $X-A$ is either finite or $\emptyset$, therefore countable. The complement of $X-A$ is A, which is also a countable set in $\Sigma$.
        \item If $A_1,A_2,...$ are in $\Sigma$, then their complements $X-A_1,X-A_2$, are finite or $\emptyset$, therefore countable. The union of these sets corresponds to the complement of the union $A=A_1\cup A_2\cup...$, which is also countable set in $\Sigma$.
        \item Therefore, the given set $\Sigma$ is a $\sigma$-algebra on $X$ when $X$ is finite.
    \end{itemize}
\end{itemize}
\paragraph*{c)}
\begin{itemize}
    \item (i) where $X$ is uncountable.
    \begin{itemize}
        \item Let's denote this set by $\Sigma$.
        \item Consider X as the set of real numbers. Let $U = \{1,...,n\}$ then $X - U = \mathbb{R} - \{1,...,n\}$ which is infinite, therefore it's in $\Sigma$. However, it's complement $\mathbb{R} - \{1,...,n\}$ is not in $\Sigma$ since $\mathbb{R} - (\mathbb{R} - \{1,...,n\}) = \{1,...,n\}$ which is not infinite or $\Sigma$ or $X$.
        \item Therefore, the given set $\Sigma$ is not a $\sigma$-algebra on $X$ when $X$ is uncountable.
    \end{itemize}
    \item (ii) where $X$ is countably infinite.
    \begin{itemize}
        \item Let's denote this set by $\Sigma$.
        \item Consider X as the set of national numbers. Let $U = \{1,...,n\}$ then $X - U = \mathbb{N} - \{1,...,n\}$ which is infinite, therefore it's in $\Sigma$. However, it's complement $\mathbb{N} - \{1,...,n\}$ is not in $\Sigma$ since $\mathbb{N} - (\mathbb{N} - \{1,...,n\}) = \{1,...,n\}$ which is not infinite or $\Sigma$ or $X$.
        \item Therefore, the given set $\Sigma$ is not a $\sigma$-algebra on $X$ when $X$ is countably infinite.
    \end{itemize}
    \item (iii) where $X$ is finite.
    \begin{itemize}
        \item Let's denote this set by $\Sigma$.
        \item Since $X$ is finite, $X-U$ will also be finite. Therefore $\Sigma$ will only include $X$ itself and the $\emptyset$.
        \item $X$ is in $\Sigma$.
        \item $\Sigma$ is closed under complementation. The complement of $X$ is the $\emptyset$ which is also in $\Sigma$ and vice versa.
        \item $\Sigma$ is closed under countable unions. The union of $X$ and $\emptyset$ is $X$ which is also in $\Sigma$.
        \item Therefore, the given set $\Sigma$ is a $\sigma$-algebra on $X$ when $X$ is finite.
    \end{itemize}
\end{itemize}
\section*{Answer 3}
\paragraph*{a)}
\begin{itemize}
            \item Let $ax \equiv b($mod $p)$ has a solution for $x$ be $s$ and $gcd(a, p)| b$ be $r$.
            \item We need to show that $s\rightarrow r$ and $r \rightarrow s$, then we're done.
            \item $s \rightarrow r$.
            \begin{itemize}
                \item Assume that $ax\equiv b($mod $ p)$, then $ax = pt + b$, $\exists t \in \mathbb{Z}$.
                \item $ax - pt = b$
                \item Let $gcd(a,p) = d$ then $d|a$ and $d|p$.
                \item We can also say that $d|ax$ and $d|pt$ $\rightarrow$ $d|(ax-pt)$.
                \item $d|(ax-pt)$ $\rightarrow$ $d|b$.
                \item We get $gcd(a,p)|b$. $s \rightarrow r$ has been proved.
            \end{itemize}{}
            \item $r \rightarrow s$.
            \begin{itemize}
                \item Assume that $d=gcd(a,p)$, then $\exists k,t \in \mathbb{Z}$  such that $ak+pt=d$ by Bezout's Identity.
                \item $b=cd \rightarrow b=c(ak+pt)=a(ck)+p(ct)$
                \item Therefore, we get a solution for $ax \equiv b($mod $p)$.
                \item $r\rightarrow s$ has been proved.
            \end{itemize}{}
\end{itemize}{}
\paragraph*{b)}
\begin{itemize}
    \item Assume we have the pair of congruences:
    \begin{equation*}
        \begin{split}
            &a_1x \equiv b_1 \pmod{p_1} \\
            &a_2x \equiv b_2 \pmod{p_2}
        \end{split}
    \end{equation*}
    \item We are given the conditions $\gcd(p_1, p_2) = 1$, $\gcd(a_1, p_1) \mid b_1$, and $\gcd(a_2, p_2) \mid b_2$.
    \item Let $d = \gcd(p_1, p_2)$. Since $d = \gcd(p_1, p_2) = 1$, we can apply Bezout's Identity to find integers $m$ and $n$ such that $mp_1 + np_2 = 1$.
    \item Now, consider the following linear combination:
    \begin{equation*}
        \begin{split}
            c = m \cdot p_1 \cdot b_2 + n \cdot p_2 \cdot b_1
        \end{split}
    \end{equation*}
    \item By rearranging the terms, we get:
    \begin{equation*}
        \begin{split}
            c &\equiv m \cdot p_1 \cdot b_2 \pmod{p_1} \\
            c &\equiv n \cdot p_2 \cdot b_1 \pmod{p_2}
        \end{split}
    \end{equation*}
    \item Now, let's consider $a = a_1 \cdot p_2 \cdot b_2 + a_2 \cdot p_1 \cdot b_1$. Notice that $a$ is a linear combination of $a_1$ and $a_2$ with coefficients being multiples of $p_1$ and $p_2$.
    \item Now, let's show that $x = c$ satisfies both congruences:
    \begin{equation*}
        \begin{split}
            a_1 \cdot x &= a_1 \cdot (m \cdot p_1 \cdot b_2 + n \cdot p_2 \cdot b_1) \\
            a_2 \cdot x &= a_2 \cdot (m \cdot p_1 \cdot b_2 + n \cdot p_2 \cdot b_1)
        \end{split}
    \end{equation*}
    \item Now, consider these expressions modulo $p_1$ and $p_2$. It will be found that they are congruent to $b_1$ and $b_2$ modulo $p_1$ and $p_2$ respectively.
    \item Therefore, $x = c$ is a solution to the system of congruences $a_1x \equiv b_1 \pmod{p_1}$ and $a_2x \equiv b_2 \pmod{p_2}$ when $\gcd(p_1, p_2) = 1$, $\gcd(a_1, p_1) \mid b_1$, and $\gcd(a_2, p_2) \mid b_2$.
\end{itemize}{}
\paragraph*{c)}
\begin{itemize}
    \item To prove that the given system of congruences has a solution of the form $x \equiv c \pmod{\Pi}$, where $\Pi = p_1p_2 \dots p_k$, $\gcd(p_1, \ldots, p_k) = 1$, and $\gcd(a_i, p_i) \mid b_i$ for some $c \in \mathbb{Z}$ and $i = 1, \ldots, k$, we can use the Chinese Remainder Theorem.
    \item The Chinese Remainder Theorem states that if $m_1, m_2, \ldots, m_k$ are pairwise coprime integers (i.e., $\gcd(m_i, m_j) = 1$ for all $i \neq j$), and $a_1, a_2, \ldots, a_k$ are any integers, then the system of simultaneous congruences:
    \begin{equation*}
        \begin{split}
            x &\equiv a_1 \pmod{m_1} \\
            x &\equiv a_2 \pmod{m_2} \\
            & \vdots \\
            x &\equiv a_k \pmod{m_k}
        \end{split}
    \end{equation*}
    has a unique solution modulo $M = m_1m_2 \dots m_k$.
    \item Now, let's relate this to the given system of congruences:
    \begin{equation*}
        \begin{split}
            a_1x &\equiv b_1 \pmod{p_1} \\
            a_2x &\equiv b_2 \pmod{p_2} \\
            & \vdots \\
            a_kx &\equiv b_k \pmod{p_k}
        \end{split}
    \end{equation*}
    \item Note that the conditions $\gcd(p_1, \ldots, p_k) = 1$ and $\gcd(a_i, p_i) \mid b_i$ for each $i$ are satisfied.
    \item Now, set $m_1 = p_1, m_2 = p_2, \ldots, m_k = p_k$. Since $\gcd(p_i, p_j) = 1$ for $i \neq j$, the conditions of the Chinese Remainder Theorem are met.
    \item By Chinese Remainder Theorem, the system of congruences has a unique solution $x$ modulo $M = p_1p_2 \dots p_k = \Pi$. Therefore, there exists a solution of the form $x \equiv c \pmod{\Pi}$.
\end{itemize}{}

\section*{Answer 4}
\paragraph*{a)}
\begin{itemize}
            \item Let's denote the set $\prod\nolimits_{i \in \mathbb{Z}^+} X$ as $X^T$. Then we'll assume that this set is countable. Since it's countable, there should exist such a function $f: \mathbb{Z}^+ \rightarrow X^T$ which is surjective.
            \item For a defined function $f$, we have $f(n) = (x_{n1}, x_{n2}, ..., x_{nn}, ...)$ where each $x_{ij} \in \{a,...,z\}$.
            \item Then we can consider $y = (y_1, y_2, ...) \in X^T$ defined by
            $$
                y_n=
                \begin{cases}
                    c \quad \text{if } x_{nn} \neq c\\
                    t \quad \text{if } x_{nn} = c\\
                \end{cases}
            $$
            \item Such defined y is not mapped to by our function $f$, it differs from each $f(n)$ by at least one coordinate. Therefore, $f$ is not surjective, we get a contradiction. The set $X^T$ is not countable.

\end{itemize}{}
\paragraph*{b)}
\begin{itemize}
    \item Let $\{Y_i\}_{i \in \mathbb{Z}^+}$ be a family of sets, each of which is countably infinite. We want to determine whether the set $S = \bigcup_{i \in \mathbb{Z}^+} Y_i$ is countable or not.
    \item To show whether $S$ is countable or not, we need to check if we can construct a function $f: \mathbb{Z}^+ \rightarrow S$ which is surjective.
    \item Since each $Y_i$ is countably infinite, we can list its elements as $Y_i = \{y_{i1}, y_{i2}, y_{i3}, ...\}$. Now, we can create a function that maps the elements of each set:
    \begin{equation*}
        \begin{split}
            f(1) = y_{11},\\
            f(2) = y_{21},\\
            f(3) = y_{12},\\
            f(4) = y_{31},\\
            f(5) = y_{22},\\
            f(6) = y_{13},\\
        \end{split}
    \end{equation*}
    and so on.
    \item Therefore, by using Cantor's diagonal argument, we can say that $S = \bigcup_{i \in \mathbb{Z}^+} Y_i$ is countable.
\end{itemize}{}
\end{document}
